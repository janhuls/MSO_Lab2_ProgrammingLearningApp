\documentclass[11pt,a4paper]{article}

%============================%
%        BASIC PACKAGES      %
%============================%
\usepackage[a4paper,margin=3cm]{geometry}
\usepackage[T1]{fontenc}
\usepackage[utf8]{inputenc}
\usepackage[english]{babel}
\usepackage{lmodern}          % Better serif font
\usepackage{microtype}        % Better spacing and justification
\usepackage{parskip}          % Paragraph spacing instead of indentation
\usepackage{setspace}
\usepackage{hyperref}
\usepackage{datetime}
\newdateformat{monthyear}{\monthname[\THEMONTH] \THEYEAR}
\date{\monthyear\today}
\usepackage{graphicx}
\usepackage{comment}

%============================%
%        HEADINGS            %
%============================%
\usepackage{titlesec}
\renewcommand{\thesection}{\arabic{section}}    %all arabic is standard
\renewcommand{\thesubsection}{\arabic{section}.\arabic{subsection}}

% enumerate settings
\usepackage{enumitem}                           % for the [resume] option
\renewcommand{\theenumi}{\arabic{enumi}}        %options are {arabic, alph, Alph, roman, Roman}
\renewcommand{\theenumii}{\alph{enumii}}
\renewcommand{\theenumiii}{\roman{enumiii}}

%============================%
%        MATHEMATICS         %
%============================%
\usepackage{amsmath,amssymb,amsthm}
\usepackage{dsfont}
\renewcommand{\P}{\mathbb{P}}
\newcommand{\N}{\mathds{N}}
\newcommand{\Z}{\mathds{Z}}
\newcommand{\Q}{\mathds{Q}}
\newcommand{\R}{\mathds{R}}
\newcommand{\C}{\mathds{C}}
\newcommand{\EX}{\mathbb{E}}
\newcommand{\PP}{\mathcal{P}}           % power set
\newcommand{\ndiv}{\nmid}
\newcommand{\eps}{\varepsilon}
\let\temp\phi                           % varphi to phi
\let\phi\varphi
\let\varphi\temp
%different set notations
\newcommand{\sub}{\subset}              % new command to refer to standard notation \subset for standard subset (which might be equal)
\newcommand{\subneq}{\subsetneq}        % new command to refer to standard notation \subsetneq for proper subset (not allowed to be equal)
%\renewcommand{\sub}{\subseteq}         % use \subseteq instead of \subset for standard subset (which might be equal)
%\renewcommand{\subneq}{\subset}        % use \subset instead of \subsetneq for proper subset (which is not allowed to be equal)
%\renewcommand{\setminus}{-}            % use - instead of \ for relative complement of sets
%\renewcommand{\neg}{{\sim}}            % use ~ as negation symbol
\newcommand\norm[1]{\lVert#1\rVert}     % norm with double vertical lines
\newcommand\normx[1]{\Vert#1\Vert}
\newcommand{\bigO}{\mathcal{O}}         % big O notation

%============================%
%        CODE BLOCKS         %
%============================%
\usepackage{minted}
\usemintedstyle{perldoc}
\setminted[haskell]{
    fontsize=\small,
    baselinestretch=1.05,
    frame=leftline,
    rulecolor=\color{black!70},
    framesep=2mm,
    linenos=false,
    tabsize=2,
    breaklines=true,
    autogobble=true
}

%============================%
%         TITLE INFO         %
%============================%
\title{\textbf{MSO \\ Loopa}}
\author{Jan Huls (4699610), Arwin Moormans (4965957)}


\begin{document} 
    \maketitle
    \begin{figure*}[htbp]
        \centering
        \includegraphics[width=0.9\textwidth]{mascot}
        \label{image-mascot}
    \end{figure*}
    \newpage
    \section*{Design}
    \subsection*{Class Diagram}
    \subsection*{Design Patterns}
    \subsection*{Deviations From Previous Design}
    To implement the \texttt{ConditionalRepeat} class, we had to make a few changes, since we encountered problems with our program needing to paste all the moves it has completed. 
    Previously we implemented a \texttt{ToString} method for all classes implementing \texttt{ICommand}, and while executing the commands, we added them to a list, and after executing we called \texttt{ToString} on them.
    This worked fine for the old commands, but the conditional repeat needs to check a character state to see how many times it should do its code block, which was not possible with our old implementation. 
    To fix this we added a list of executed commands to the character, and each basic command (Turn and Move) adds itself to this list.
    This way we can easily print the list in the \texttt{Character} class. 
    
    \section{Code Quality}
    \subsection{Resharper}
    We use the Rider IDE from Jetbrains as our main IDE for coding this project, luckely Rider has a great build-in code Quality tool called Resharper.
    Resharper provides good feedback on code quality and we have used it intensifly throughout the whole project.
    Resharper gives errors (wich are actual c# errors) and warnings wich is feedback on our code quality.
    These warnings can be naming conventions, possible null values, redundend code and much more.
    This way we always wrote code that complies to the code quality standards
    \subsection{Metrices}
    kkr blijkbaar moet dit ook
    \subsection{Reflection + Code reviews}
    Using Resharper took a bit of time getting used to.
    Normaly we didn't really care about, naming conventions or possible null values and only cared about that the code worked.
    But now we needed to, and it actualy made the project a lot more well-organized. 
    Throughout the project we frequently took a look at each others code and gave feedback on each others work.
    Giving each other ideas of how the code could be improved or refactored, or making sure that we both understood each others code.
    By using Resharper and checking on eachother, we have writen clean code that complies to the code quality standards.

    \section{Evaluation}
    \subsection{Analysis}
    We strive of cource for high \texttt{Cohesion} and low \texttt{Coupling} to make the design of our code as good as possible.
    To achieve low \texttt{Coupling} we have seperated the project in 3 sub projects.
    \begin{enumerate}
        \item MSOProgramLearningApp
        This project holds the core logic of our application and holds all the data structures such as the character, grid and commands.
        It also is responsible for parsing strings into commands, moving the character and drawing an output image.
        This project on its own is just the commandline interface we made at the previous practical assigment
        \item MSOAvaloniaApp
        This projects uses the Avalonia libary to create a window for the actual finished application.
        All the UI related code is in here, and it uses a reference to the \texttt{MSOProgramLearningApp} project for all the core logic
        \item MSOTestProject
        This project is used for testing the logic in the \texttt{MSOProgramLearningApp} project.
        It holds unit and system tests that we will discus in the next section
    \end{enumerate}
    This way we also make sure that each project has high \texttt{Cohesion} because it only does a specific part of our application.
    %moeten we nog dieper op cohesion and coupling in gaan? als in echt per class kijken waarom hij eraan voldoet of is dit prima?
    \subsection{Changing Requirements}

    \section{Testing}
    \subsection{Unit tests}
    \subsection{System tests} % moet nog
    \subsection{Final test run} %foto van alle geslaagde tests op het eind

    \section{Work Distribution & Retrospective}
    \subsection{Task Distribution}
    \subsection{Retrospective}

    %begint altijd (0, 0)
    %grid is altijd vierkant
    %geen oneindige grids
    %kan niet uit grid lopen??
    
\end{document}